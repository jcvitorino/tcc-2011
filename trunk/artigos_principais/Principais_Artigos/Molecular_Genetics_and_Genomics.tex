\chapter{Molecular Genetics and Genomics}

{\Large \bf Characterization of the TaAIDFa gene encoding a CRT/DRE-binding factor responsive to drought, high-salt, and cold stress in wheat}

Zhao-Shi Xu, Zhi-Yong Ni, Li Liu, Li-Na Nie, Lian-Cheng Li, Ming Chen, You-Zhi Ma

{\it Molecular Genetics and Genomics}, Vol. 280, No. 6. (1 December 2008), pp. 497-508. doi:10.1007/s00438-008-0382-x

{\large \bf Abstract}

Dehydration responsive element-binding factors (DBFs) belong to the AP2/ERF superfamily and play vital regulatory roles in abiotic stress responses in plants. In this study, we isolated three novel homologs of the DBF gene family in wheat (Triticum aestivum L.) by screening a drought-induced cDNA library and designated them as TaAIDFs (T. aestivum abiotic stress-induced DBFs). Compared to TaAIDFb and TaAIDFc, TaAIDFa lacks a short Ser/Thr-rich region, a putative phosphorylation site, following the AP2/ERF domain. The TaAIDFa gene, located on chromosome 3BS, is interrupted by a single intron at the 17th Arg (R) in the N-terminal domain. The N-terminal region of the TaAIDFa protein modulates nuclear localization. The TaAIDFa protein is capable of binding to CRT/DRE elements in vitro and in vivo, and of trans-activating reporter gene expression in yeast cells. The TaAIDFa promoter, with various stress-related cis-acting elements, drives expression of the GUS reporter gene in wheat calli under stress conditions. This was further confirmed by responses of TaAIDFa transcripts to drought, salinity, low-temperature, and exogenous ABA. Furthermore, overexpression of TaAIDFa activated CRT/DRE-containing genes under normal growth conditions, and improved drought and osmotic stress tolerances in transgenic Arabidopsis plants. These results suggested that TaAIDFa encodes a CRT/DRE element-binding factor that might be involved in multiple abiotic stress signal transduction pathways.