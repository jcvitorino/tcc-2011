\chapter{The Plant Journal}


{\Large \bf Research on plant abiotic stress responses in the post-genome era: past, present and future}

Takashi Hirayama e Kazuo Shinozaki

{\it The Plant Journal}, Vol. 61, No. 6. (March 2010), pp. 1041-1052. doi:10.1111/j.1365-313X.2010.04124.x

{\large \bf Abstract}

Understanding abiotic stress responses in plants is an important and challenging topic in plant research. Physiological and molecular biological analyses have allowed us to draw a picture of abiotic stress responses in various plants, and determination of the Arabidopsis genome sequence has had a great impact on this research field. The availability of the complete genome sequence has facilitated access to essential information for all genes, e.g. gene products and their function, transcript levels, putative cis-regulatory elements, and alternative splicing patterns. These data have been obtained from comprehensive transcriptome analyses and studies using full-length cDNA collections and T-DNA- or transposon-tagged mutant lines, which were also enhanced by genome sequence information. Moreover, studies on novel regulatory mechanisms involving use of small RNA molecules, chromatin modulation and genomic DNA modification have enabled us to recognize that plants have evolved complicated and sophisticated systems in response to complex abiotic stresses. Integrated data obtained with various 'omics' approaches have provided a more comprehensive picture of abiotic stress responses. In addition, research on stress responses in various plant species other than Arabidopsis has increased our knowledge regarding the mechanisms of plant stress tolerance in nature. Based on this progress, improvements in crop stress tolerance have been attempted by means of gene transfer and marker-assisted breeding. In this review, we summarize recent progress in abiotic stress studies, especially in the post-genomic era, and offer new perspectives on research directions for the next decade.