\chapter{Genomic Signal Processing and Statistics (GENSIPS)}


{\Large \bf Transcription Factor Discovery using Support Vector Machines and Heterogeneous Data}

Barbe, J.F.;   Tewfik, A.H.;   Khodursky, A.B.;  

(June 2007), pp. 1-4. doi:10.1109/GENSIPS.2007.4365812 Key: Barbe2007

{\large \bf Abstract}

In this work we analyze the suitability of expression and sequence data for discovery of co-regulatory relationships using Support Vector Machines. In addition, we try to assess the possibility of improving such results by heterogeneous data fusion and by estimating a probability of a correct classification. As shown in other studies, we have found that transcription co-expression is a good estimator for genetic co-regulation. We also have found some evidence that operator site sequence motifs can be used to estimate co-regulation, but the kernels used for feature extraction did not achieve classification rates comparable to expression data. Finally, the additional information provided by combining sequence and expression data can be exploited to estimate the probability of correct classification.