\chapter{International Conference on Machine Learning and Applications (ICMLA)}

{\Large \bf SVMotif: A Machine Learning Motif Algorithm}

Mark Kon, Dustin T Holloway e Charles DeLisi

{\it Machine Learning and Applications}, 2007. ICMLA 2007. Sixth International Conference on (15 December 2007), pp. 573-580. doi:10.1109/ICMLA.2007.105

{\large \bf Abstract}

We describe SVMotif, a support vector machine-based learning algorithm for identification of cellular DNA transcription factor (TF) motifs extrapolated from known TF-gene interactions. An important aspect of this procedure is its ability to utilize negative target information (examples of likely non-targets) as well as positive information. Applications involve situations where clusters of genes are distinguished in experiments with known transcription factors without known binding locations. We apply this to yeast TF data with target identifications from ChlP-chip and other sources, and compare performance with Gibbs sampling methods such as BioProspector. We verify that in yeast this method implies well-defined and cross-validated statistical correlations between TF binding and secondary motifs whose binding properties (either with the primary TF or other possible promoters) are not certain, and discuss some implications of this. SVMotif can be a useful standalone method or a complement to existing techniques, and it will be made publicly available.