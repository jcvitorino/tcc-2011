\chapter{Current Opinion in Plant Biology}


{\Large \bf cis-Regulatory elements in plant cell signaling}

Henry D Priest, Sergei A Filichkin e Todd C Mockler

{\it Current Opinion in Plant Biology}, Vol. 12, No. 5. (01 October 2009), pp. 643-649. doi:10.1016/j.pbi.2009.07.016

{\large \bf Abstract}

Plant cell signaling pathways are in part dependent on transcriptional regulatory networks comprising circuits of transcription factors (TFs) and regulatory DNA elements that control the expression of target genes. Here, we describe experimental and bioinformatic approaches for identifying potential cis-regulatory elements. We also discuss recent integrative genomics studies aimed at elucidating the functions of cis-regulatory elements in aspects of plant biology, including the circadian clock, interactions with the environment, stress responses, and regulation of growth and development by phytohormones. Finally, we discuss emerging technologies and approaches that offer great potential for accelerating the discovery and functional characterization of cis-elements and interacting TFs--which will help realize the promise of systems biology.



{\Large \bf 'Omics' analyses of regulatory networks in plant abiotic stress responses}

Kaoru Urano, Yukio Kurihara, Motoaki Seki, Kazuo Shinozaki

{\it Current Opinion in Plant Biology}, Vol. 13, No. 2. (April 2010), pp. 132-138. doi:10.1016/j.pbi.2009.12.006

{\large \bf Abstract}

Plants must respond and adapt to abiotic stresses to survive in various environmental conditions. Plants have acquired various stress tolerance mechanisms, which are different processes involving physiological and biochemical changes that result in adaptive or morphological changes. Recent advances in genome-wide analyses have revealed complex regulatory networks that control global gene expression, protein modification, and metabolite composition. Genetic regulation and epigenetic regulation, including changes in nucleosome distribution, histone modification, DNA methylation, and npcRNAs (non-protein-coding RNA) play important roles in abiotic stress gene networks. Transcriptomics, metabolomics, bioinformatics, and high-through-put DNA sequencing have enabled active analyses of regulatory networks that control abiotic stress responses. Such analyses have markedly increased our understanding of global plant systems in responses and adaptation to stress conditions.