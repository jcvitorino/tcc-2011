\chapter{Computatinal Biology and Chemistry}
{\Large \bf The cross-species prediction of bacterial promoters using a support vector machine}

Michael Towsey, Peter Timms, James Hogan e Sarah A. Mathews

{\it Computational Biology and Chemistry}, Vol. 32, No. 5. (October 2008), pp. 359-366. doi:10.1016/j.compbiolchem.2008.07.009

{\large \bf Abstract}

Due to degeneracy of the observed binding sites, the in silico prediction of bacterial sigma 70-like promoters remains a challenging problem. A large number of  sigma 70-like promoters has been biologically identified in only two species, Escherichia coli and Bacillus subtilis. In this paper we investigate the issues that arise when searching for promoters in other species using an ensemble of SVM classifiers trained on E. coli promoters. DNA sequences are represented using a tagged mismatch string kernel. The major benefit of our approach is that it does not require a prior definition of the typical −35 and −10 hexamers. This gives the SVM classifiers the freedom to discover other features relevant to the prediction of promoters. We use our approach to predict sigma A promoters in B. subtilis and sigma 66 promoters in Chlamydia trachomatis. We extended the analysis to identify specific regulatory features of gene sets in C. trachomatis having different expression profiles. We found a strong −35 hexamer and TGN/−10 associated with a set of early expressed genes. Our analysis highlights the advantage of using TSS-PREDICT as a starting point for predicting promoters in species where few are known.